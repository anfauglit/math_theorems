\documentclass[letterpaper, 10pt]{article}
\usepackage{amsmath, amsthm, amssymb}
\usepackage{mathtools}
\usepackage{enumitem}
\usepackage{listings}

\DeclarePairedDelimiter\setsize{\lvert}{\rvert}
\DeclarePairedDelimiter\norm{\lVert}{\rVert}

\setlength{\parindent}{0pt}

\newtheorem{thm}{Theorem}
\newtheorem{lem}[thm]{Lemma}
\newtheorem{col}[thm]{Corollary}

\theoremstyle{definition}
\newtheorem{mydef}[thm]{Definition}

\begin{document}
\section{Mappings}
\begin{lem}
	If $f\colon S \to T$, $g\colon T \to S$ are any mappings such that
	\begin{equation*}
		g \circ f = 1_S
	\end{equation*}
	then $f$ is injective and $g$ is surjective.
\end{lem}

\begin{thm}
	A mapping $f\colon S \to T$ is a bijection if and only if there is a
	mapping $g \colon T \to S$ to satisfy the following:
	\begin{align*}
		g \circ f = 1_S && f \circ g = 1_T
	\end{align*}
\end{thm}

\begin{thm}
	If $f \colon A \to B$ and $g \colon B \to C$ are both injective then $g
	\circ f$ is injective.
\end{thm}

\begin{thm}
	If $f \colon A \to B$ and $g \colon B \to C$ are both surjective then $g
	\circ f$ is surjective.
\end{thm}

\begin{thm}
	If $f \colon A \to B$, $g \colon B \to C$ and $g \circ f$ is surjective
	then $g$ is surjective. 
\end{thm}

\begin{thm}
	If $f \colon A \to B$, $g \colon B \to C$ and $g \circ f$ is injective 
	then $f$ is injective. 
\end{thm}

\begin{mydef}
	A set $X$ is \textbf{finite} if and only if there is one-to-one function $f\colon X \to n$ for some natural number $n$.
\end{mydef}

\begin{mydef}
	Suppose that $X$ is a finite set. Let $n$ be the least natural number such that there is a one-to-one function $f\colon X \to n$.
	Then $n$ is the \textbf{number of elements} in the set $X$, and we write $\setsize{X} = n$.
\end{mydef}

\begin{thm}
	Let A be finite and let $n = \setsize{A}. If $f\colon A \to n$ is one-to-one, then $f$ is onto $n$.
\end{thm}

\begin{col}
	If $A$ is finite and $n = \setsize{A}$, then there is a bijection $f\colon A \to n$.
\end{col}

\begin{thm}
	Let $f\colon A \to A$ be one-to-one where $A$ is a finite set. Then $f$ is onto $A$.
\end{thm}

\end{document}
