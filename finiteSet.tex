\documentclass[letterpaper, 10pt]{article}
\usepackage{amsmath, amsthm, amssymb}
\usepackage{mathtools}
\usepackage{enumitem}
\usepackage{listrings}

\DeclarePairedDelimiter\setsize{\lvert}{\rvert}
\DeclarePairedDelimiter\norm{\lVert}{\rVert}

\setlength{\parindent}{0pt}

\newtheorem*{thm*}{Theorem}[section]
\numberwithin{thm}{subsection}
\newtheorem{thm}{Theorem}
\newtheorem{lem}[thm]{Lemma}
\newtheorem{col}[thm]{Corollary}

\theoremstyle{definition}
\newtheorem{mydef}[thm]{Definition}

\begin{document}
\begin{mydef}
	A set $X$ is \textbf{finite} if and only if there is one-to-one function $f\colon X \to n$ for some natural number $n$.
\end{mydef}

\begin{mydef}
	Suppose that $X$ is a finite set. Let $n$ be the least natural number such that there is a one-to-one function $f\colon X \to n$.
	Then $n$ is the \textbf{number of elements} in the set $X$, and we write $\setsize{X} = n$.
\end{mydef}

\begin{thm}
	Let A be finite and let $n = \setsize{A}. If $f\colon A \to n$ is one-to-one, then $f$ is onto $n$.
\end{thm}

\end{document}

